%%%%%%%%%%%%%%%%%%%%%%%%%%%%%%%%%%%% Chapter Template

\chapter{Introdução} 	% Main chapter title
\label{Chapter1} 		% For referencing the chapter elsewhere, usage \ref{Chapter1}

%%%%%%%%%%%%%%%%%%%%%%%%%%%%%%%%%%%%

Neste capitulo será apresentado o problema a ser resolvido, a descrição do projeto e os seus objetivos, além disso, será apresentado as atividades desenvolvidas em etapas ao longo do projeto e por fim será descrito como o documento foi dividido.
	

%%%%%%%%%%%%%%%%%%%%%%%%%%%%%%%%%%%% SECTION 1

\section{Contextualização}
\label{sec:Ch1.1}

A abordagem da arquitetura de microsserviços no desenvolvimento de \textit{software} tem cada vez ganho, mais distinção nos últimos anos. Surge pelos problemas registados pelas empresas na construção de um sistema que seja escalável e flexível relativamente aos requisitos de hoje em dia. A construção de um sistema numa só aplicação dificulta a implementação desses mesmos aspetos.

Cada serviço é responsável pela execução de uma tarefa específica ou um grupo específico de tarefas, agrupados conforme as funcionalidades. Visto que, que cada serviço pode funcionar de uma forma independente em relação aos outros serviços, estes podem ser escalado individualmente, fazendo com que, o desenvolvimento e manutenção do sistema seja mais ágil e flexível.

A arquitetura de microsserviços facilita a integração com outros sistemas e serviços, mas, por outro lado,
é preciso garantir que a comunicação entre eles seja bem-sucedido, isso inclui a sua segurança, disponibilidade e escalabilidade, caraterísticas importantes para uma comunicação eficiente.

As novas linguagens de programação como \textit{Ballerina} fornecem mecanismos mais simplificados e eficientes na implementação de microsserviços e disponibilização suporte nativos à maioria dos protocolos de comunicação (\textit{HTTP}, \textit{RPC}, \textit{\ac{imap}}, entre outros), que permitem aos desenvolvedores resolver muitos dos problemas referidos em cima. Além disso, recentemente chegou ao top 100 das linguagens de programação mais populares, mais em específico coloca-se no lugar 87º do \textit{ranking}, ou seja, indica que cada vez mais, esta linguagem está a ganhar mais popularidade no tema em estudo \cite{top100}. 

%%%%%%%%%%%%%%%%%%%%%%%%%%%%%%%%%%%% SECTION 2

\section{Descrição do Projeto}
\label{sec:Ch1.2}

É importante mencionar que este trabalho é de natureza académica e visa explorar as possibilidades da linguagem de programação para um público amplo, incluindo aqueles que não possuem ampla experiência no desenvolvimento de aplicações com múltiplos serviços. Além disso, todo o código desenvolvido e o presente documento será disponibilizado publicamente no \textit{Github}, com a licença X (também chamada licença MIT).


%%%%%%%%%%%%%%%%%%%%%%%%%%%%%%%%%%%% SUBSECTION 1

\subsection{Objetivos}
\label{sub:Ch1.2.1}

O trabalho passa por desenvolver uma aplicação com a linguagem de programação \textit{Ballerina} com múltiplos serviços (\textit{Sandwich}, \textit{Ingredient}, \textit{Review}, \textit{Shop}, \textit{Order}, \textit{Customer}) para suprir as necessidades de uma loja de sanduíches com funcionalidades relacionadas com o negócio, como, por exemplo, criação de sanduíches, listagem dos ingredientes disponíveis, realizar uma encomenda, com algum suporte multilíngua. Uma arquitetura baseada em microsserviços será adotada recorrendo também a eventos para comunicação entre serviços.
Todo o trabalho ficará disponível num repositório público como contributo para divulgação da linguagem, das suas características, potenciais dificuldades, tal como, problemas que condicionem a sua adoção.

O objetivo deste trabalho é desenvolver todas as funcionalidades que suportem as operações necessárias para o funcionamento correto do sistema na totalidade com duas restrições: a linguagem de programação tem de ser \textit{Ballerina} e a aplicação deve usar todas as características próprias da linguagem para a implementação de microsserviços. Dado a implementação das funcionalidades anteriormente mencionadas, devem ser feitos, os testes necessários, testes unitários como testes de desempenho para garantir a boa implementação das regras de negócio.

%%%%%%%%%%%%%%%%%%%%%%%%%%%%%%%%%%%% SECTION 3

\section{Calendarização}
\label{sec:Ch1.3}




%%%%%%%%%%%%%%%%%%%%%%%%%%%%%%%%%%%% SECTION 4

\section{Organização do Relatório}
\label{sec:Ch1.4}


Este relatório apresenta-se dividido nos seguintes capítulos:

\begin{itemize}
    \item \textbf{Introdução}: Introdução ao problema e descrição do mesmo
    \item \textbf{Introdução teórica}: Introdução aos tipos de arquitetura, comunicações entre serviços e estudo sobre as tecnologias
    \item \textbf{\textit{Ballerina}}: Estudo sobre a linguagem de programação \textit{Ballerina}
    \item \textbf{Deteção de linguagem}: Estudo sobre aplicações de deteção de linguagem
    \item \textbf{Implementação}: Apresentação da implementação da aplicação com vários serviços
    \item \textbf{Conclusão}: Conclui a apresentação do tema, apresenta trabalho futuro e descrição dos resultados obtidos
\end{itemize}



