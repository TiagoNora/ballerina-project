%%%%%%%%%%%%%%%%%%%%%%%%%%%%%%%%%%%% Chapter Template

\chapter{Conclusões} 	% Main chapter title
\label{Chapter7} 		% For referencing the chapter elsewhere, usage \ref{Chapter6}

%%%%%%%%%%%%%%%%%%%%%%%%%%%%%%%%%%%%


%%%%%%%%%%%%%%%%%%%%%%%%%%%%%%%%%%%%

\section{Resultados}
O desenvolvimento de microsserviços com a linguagem em estudo foi conseguida, tendo sido, possível explorar muitas das suas características. A linguagem proporcionou uma experiência agradável e eficiente na construção de microsserviços e permitiu a criação de um sistema flexível e altamente modularizado.

\section{Dificuldades}
O desenvolvimento e implementação dos conteúdos apresentados nas secções anteriores trouxe algumas dificuldades no que diz respeito a informação, devido a Ballerina ser uma linguagem recente e ainda ter uma comunidade pequena, a documentação disponibiliza pelos criadores é insuficiente para alguns dos casos e muitas vezes as respostas encontradas para perguntas realizadas em fóruns comunitárias é desatualizada devido à constante atualização da linguagem.

\section{Trabalho Futuro}
No futuro seria interessante existir o desenvolvimento do \textit{front end} da aplicação, ser feito a implementação do software para a gestão de pagamentos e \textit{stock}, completando assim o sistema já desenvolvido e torna-o num produto pronto para produção, ou seja, estar a ser usado no dia a dia de alguma empresa.



\label{sec:Ch6.1}






