%----------------------------------------------------------------------------------------
%	ABSTRACT PAGES
%----------------------------------------------------------------------------------------

% IMPORTANT NOTE: the abstract must always be written in two languages. If the report
% is written in Portuguese you have selected 'portuguese' as the language in the document class.
% Therefore, the portuguese version of the abstract must come first, so write it in the
% below area denoted by 'MAIN LANGUAGE ABSTRACT'. The english version follows in the
% 'SECOND LANGUAGE ABSTRACT' section.
% If the report is written in English, first will come the abstract in English
% ('MAIN LANGUAGE ABSTRACT') and then in Portuguese ('SECOND LANGUAGE ABSTRACT').

\begin{abstract}
%%%%%%%%%%%%%%%%%%%%%%%%%%%%%% MAIN LANGUAGE ABSTRACT %%%%%%%%%%%%%%%%%%%%%%%%%%%%%%%%%%

Os microsserviços são uma abordagem que atualmente tem cada vez mais ganho popularidade e relevância. Tem ganho este prestigio e adoção devido as suas caraterísticas comparando com outras arquiteturas.

Hoje em dia, esta arquitetura é muito utilizada devido ao crescente número de utilizadores, aumento de tráfego necessário e diferentes localizações desses mesmos utilizadores.

Esta abordagem é vantajosa dado que cada um dos serviços pode ser implementado e testado independentemente, permitindo assim o desenvolvimento paralelo, facilidade na manutenção do sistema e adição de novas funcionalidades.

O uso de microsserviços auxiliado com o uso de ferramentas de gestão de containers facilitam a implementação do serviço e promovem caraterísticas já existentes como, por exemplo, a escalabilidade e elasticidade.

Apesar das vantagens, esta arquitetura apresenta desafios, tais como, na comunicação entre serviços e à medida que são adicionados mais serviços a complexidade do sistema aumenta.

No presente documento será explorado a linguagem em estudo, Ballerina, como são implementados os serviços e as várias tecnologias associadas ao desenvolvimento de um sistema em que é usado microsserviços. 

%----------------------------------------------------------------------------------------

\vspace*{10mm} 
\noindent
\textbf{\keywordslabel}: microsserviços, Ballerina, serviços web, deteção de linguagem

%%%%%%%%%%%%%%%%%%%%%%%%% END OF THE MAIN LANGUAGE ABSTRACT %%%%%%%%%%%%%%%%%%%%%%%%%%%%%%
\end{abstract}
\begin{secondlangabstract}
%%%%%%%%%%%%%%%%%%%%%%%%%%%%%% SECOND LANGUAGE ABSTRACT %%%%%%%%%%%%%%%%%%%%%%%%%%%%%%%%%%

Microservices are an approach that is currently gaining more and more popularity and relevance. It has gained this prestige and adoption due to its characteristics when compared to other architectures.

Today, this architecture is widely used due to the growing number of users, the increase in traffic required, and the different locations of those users.

This approach is advantageous because each of the services can be implemented and tested independently, thus allowing parallel development, easy system maintenance, and the addition of new features.

The use of microservices aided with the use of container management tools facilitates service implementation and promotes existing features such as scalability and elasticity.

Despite the advantages, this architecture presents challenges, such as in the communication between services and as more services are added the complexity of the system increases.

This paper will explore the language under study, Ballerina, how the services are implemented and the various technologies associated with the development of a system in which microservices are used. 


%----------------------------------------------------------------------------------------

\vspace*{10mm} 
\noindent
\textbf{\keywordslabel}: microservices, Ballerina, web services, language detection

%%%%%%%%%%%%%%%%%%%%%%%%%% END OF THE SECOND LANGUAGE ABSTRACT %%%%%%%%%%%%%%%%%%%%%%%%%%%%%
\end{secondlangabstract}

